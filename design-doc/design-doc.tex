\documentclass{article}
\usepackage{fullpage}
\usepackage{siunitx} % US units
    \DeclareSIUnit\in{in}
    \DeclareSIUnit\ft{ft}
    
    \DeclareSIUnit\slug{slug}
    \DeclareSIUnit\lbm{lbm}
    
    \DeclareSIUnit\lb{lbf}
    \DeclareSIUnit\lbf{lbf}
    
    \DeclareSIUnit\psi{psi}
    
    \DeclareSIUnit\fahrenheit{\SIUnitSymbolDegree F}
    \DeclareSIUnit\rankine{R}
 \usepackage{amsmath} % to get: {align} {alignat} \usepackage{amssymb}
\usepackage{graphicx}
\usepackage{wrapfig}
\usepackage{enumitem} % to specify enumeration format
\usepackage{multicol} % for lists in two columns
\usepackage{tikz}
\usetikzlibrary{arrows, shapes}
\usetikzlibrary{arrows.meta}
\graphicspath{ {./images/} }

\setlength{\parskip}{1em}

% some values in this file should be overwritten by the lines below
\newcommand{\theauthor}{Ivan Kenevich}
\newcommand{\theemail}{ik2@pdx.edu}
\newcommand{\coursecode}{ENTER COURSECODE ENTER COURSECODE}
\newcommand{\coursename}{ENTER COURSENAME ENTER COURSENAME}
\newcommand{\thetitle}{ENTER TITLE ENTER TITLE}
\newcommand{\theduedate}{ENTER DUEDATE ENTER DUEDATE}

\newcommand{\Given}{\underline{Given:}\ }
\newcommand{\Find}{\underline{Find:}\ }
\newcommand{\Solution}{\textbf{Solution: }\\ }

\newcommand{\hHg}{\ensuremath{h_{\text{Hg}}}}
\newcommand{\rhoWat}{\ensuremath{\rho_{\text{H2O}}}}
\newcommand{\rhoOil}{\ensuremath{\rho_{\text{Oil}}}}
\newcommand{\rhoHg}{\ensuremath{\rho_{\text{Hg}}}}

\newcommand{\Patm}{\ensuremath{P_{\text{atm}}}}
\newcommand{\Pabs}{\ensuremath{P_{\text{abs}}}}
\newcommand{\Pgage}{\ensuremath{P_{\text{gage}}}}

\newcommand{\Psat}{\ensuremath{P_{\text{sat}}}}
\newcommand{\Tsat}{\ensuremath{T_{\text{sat}}}}

\newcommand{\GFS}[2]{
    \begin{tabular}[t]{rl}
        \Given & #1 \vspace{0.2cm} \\
             %& something else can go here
        \Find  & #2 \vspace{0.2cm} \\
             %& something else can go here
    \end{tabular}

    \Solution

}

\newcommand{\problem}[1]{\section*{#1}}
\newcommand{\subproblem}[1]{\subsection*{\small #1}}
 
%%                       START HERE
%%
%% THIS NEEDS TO BE HERE AND SET PROPERLY
%% THE VERSION IN COMMON HEADERS WILL LOOK THE SAME AS THIS
%% =========================================================
\renewcommand{\theauthor}{Christopher Medlin, Will DeBernardi}
\renewcommand{\theemail}{cmedlin@unm.edu, wdebernardi@unm.edu}
\renewcommand{\coursecode}{CS 351L}
\renewcommand{\coursename}{Design of Large Programs}
\renewcommand{\thetitle}{Project 3 Design}
\renewcommand{\theduedate}{04 Apr 2021}
%% ==========================================================
\usepackage{fancyhdr} \setlength{\headheight}{15pt}

\pagestyle{fancy}
\fancyhf{}
\lhead{\coursecode}
\chead{\theauthor\ - \thetitle}
\rhead{\thepage}
\renewcommand{\headrulewidth}{0.4pt} % header line thickness
\setlength{\headsep}{10pt} % spacing between the header and page text (applies to 1st page as well, bad)

\begin{document}
\thispagestyle{empty}
\raggedright{
    {\Large \textbf{\coursecode} - \textit{\coursename} \par}
    {\Large \thetitle \par}
    \vspace{0.2cm}
    {\large \theauthor\ }
    {\large \texttt{\textless \theemail \textgreater} \par}
    {\theduedate}
}
\noindent\rule{\textwidth}{0.4pt}

%{\scshape\LARGE Columbidae University \par}
%\vspace{1cm}
%{\scshape\Large Final year project\par}
%\vspace{1.5cm}
%{\huge\bfseries Pigeons love doves\par}
%\vspace{2cm}
%{\Large\itshape John Birdwatch\par}
%\vfill
%supervised by\par
%Dr.~Mark \textsc{Brown}

%\vfill


%% DOCUMENT CONTENT STARTS HERE
%% ===========================================================

\tikzset{line width=10pt}
\tikzstyle{class} = [rectangle, draw,
text width=5em, text centered, rounded corners, minimum height=2em,
node distance=4cm, minimum width=7em]
\tikzstyle{object} = [rectangle, draw,
text width=5em, text centered, minimum height=3em,
node distance=4cm, minimum width=7em]
\tikzstyle{procedure} = [ellipse, draw, node distance=3cm,
minimum height=2em]
\tikzstyle{uses} = [draw, -{Latex[width=5pt, length=5pt]}]
\tikzstyle{inherits} = [draw, -{Triangle[open, width=8pt, length=8pt]}]
\tikzstyle{hasa} = [draw, -{Diamond[open, width=6pt, length=8pt]}]
\usetikzlibrary{positioning}

\section{Main Diagram}
\begin{center}
\begin{tikzpicture}
    \node[class] (Main) {Main};
    \node[object, right of=Main] (Controller) {Controller};
    \node[object, below of=Main] (Generator) {Generator};
    \node[object, below of=Generator] (Solver) {Solver};
    \node[class, right of=Controller] (Display) {Display};
    \node[object, left of=Generator] (Maze) {Maze};
    \node[object, below of=Maze] (Cell) {Cell};

    \path[uses] (Main) -- node[above] {FXML}(Controller);
    \path[inherits] (Controller) -- (Display);
    \path[uses] (Main) -- node[right] {new thread}(Generator);
    \path[uses] (Generator) -- node[right] {onGenerationComplete}(Solver);
    \path[uses] (Generator) -| node[right] {thread-safe operation}(Display);
    \path[uses] (Solver) -| (Display);
    \path[uses] (Generator) -- (Maze);
    \path[uses] (Solver) -- (Maze);
    \path[uses] (Maze) -- (Cell);
    \path[uses] (Cell) edge [loop below] node[below] {graph}(Cell);
\end{tikzpicture}
\end{center}

\section{Notes}
\begin{itemize}
    \item All maze generation and solving algorithms will extend off of Generator and Solver respectively,
    \item The method of Display to update it \textbf{must} be thread-safe/atomic, as several different threads may be calling it
          at the same time.
    \item The maze will be stored as a graph of cells, each node being linked via edges to 4 other nodes
    \item Each edge of the maze graph will be marked as a wall or not a wall.
\end{itemize}


\end{document}
